\documentclass[../main.tex]{subfiles}

\begin{document}

\chapter{Conclusões e Discussão}\label{cap:conclusoes}

Este trabalho propôs uma abordagem baseada em realidade aumentada para o desenvolvimento de apresentações multimídia. A abordagem proposta é baseada no modelo NCM, mais especificamente em seus elos causais, condições e ações. Além disso, foi apresentada a ferramenta BumbAR, que implementa a abordagem proposta com uma implementação baseada no Unity 3D, permitindo que usuários criem, visualizem, e gerem um documento NCL de suas apresentações multimídia. Para avaliar a proposta deste trabalho, foi desenvolvido um estudo qualitativo baseado no modelo TAM. Os participantes do estudo consideraram que a proposta desenvolvida é tanto útil quanto fácil de usar. Tal estudo também permitiu a evolução da proposta, evidenciando a necessidade da criação de algo que permitisse a visualização da apresentação sendo criada como um todo.

Apesar da implementação atual da ferramenta BumbAR discutida na Seção~\ref{sec:implementacao_prototipo} permitir o desenvolvimento de uma grande variedade de apresentações multimídia, ela ainda não suporta completamente todas as condições e ações presentes no modelo NCM, como as relacionadas a seleções e atribuições, por exemplo. Essas condições e ações permitem a modificação de maneira dinâmica das propriedades de objetos de mídia em uma apresentação multimídia, como seu volume e transparência. A implementação de outras condições e ações é deixada com um trabalho futuro.

Como mencionado anteriormente, a abordagem proposta utilizou os \emph{elos causais} do modelo NCM para especificar as relações e o sincronismo temporal entre objetos de mídia. Como outro trabalho futuro, pretende-se explorar outras abordagens para especificar tal sincronismo. Por exemplo, a abordagem da linguagem SMIL~\cite{rutledge2001smil} propõe o sincronismo entre objetos de mídia com base em dois conceitos: \emph{par} e \emph{seq}. Objetos de mídia em \emph{seq}(sequência) tocam sequencialmente: quando um acaba o próximo começa; enquanto objetos de mídia em \emph{par}~(paralelo) tocam simultaneamente: eles começam e terminam ao mesmo tempo. Para alguns tipos de apresentações multimídia, como uma apresentação de slides, é possível facilitar o processo de autoria utilizando tais abstrações.

É importante ressaltar que, neste trabalho, a realidade aumentada foi utilizada em apenas uma das etapas do processo da criação de apresentações multimídia: a criação de relações causais entre os objetos de mídia. É válido questionar a viabilidade da utilização desse tipo de abordagem na especificação espacial desses objetos de mídia. Nesse contexto, por exemplo, a especificação de que um objeto de mídia que ocupa um terço da altura da tela e dois quintos da largura não seria simples com a realidade aumentada. Isso se deve, principalmente, à falta de uma alta precisão na detecção de marcadores. Uma dificuldade motora do usuário também poderia prejudicar consideravelmente tal processo de especificação de objetos de mídia. 

Uma possível solução para esse tipo de dificuldade seria o uso de descritores pré-definidos para a especificação de regiões na tela. A linguagem NCL permite a criação de descritores que podem ser utilizados por diversos objetos de mídia, facilitando o reúso. Esses descritores definem a configuração espacial de objetos de mídia. No caso da realidade aumentada, poderiam existir marcadores de descritores pré-definidos que seriam, em tempo de execução, associados a objetos de mídia pelo usuário. Entretanto, a discussão resurge no que diz respeito à criação de tais descritores. A especificação das características destes utilizando a realidade aumentada recairia no problema da precisão. Caso os descritores fossem impostos pela ferramenta, não seria necessária uma preocupação de como os usuários iriam definí-los, pois já estariam pré-estabelecidos. Essa imposição, contudo, diminuiria a variedade de posições que objetos de mídia poderiam ocupar na tela. Por essas razões, a realidade aumentada pode não substituir completamente todas as etapas do processo de autoria multimídia. Entretanto, este trabalho demonstra que a RA pode trazer mais simplicidade e conveniência em algumas etapas desse processo.

Neste trabalho, não foram levados em conta aspectos ergonômicos na avaliação da proposta. A manipulação dos marcadores de realidade aumentada por um longo período poderia causar nos usuários desconforto e, em maior grau, vir a causar lesões por esforço repetitivo devido à manutenção por um longo período de posições desconfortáveis.

Algo que poderia vir a contribuir para evolução deste trabalho seria uma avaliação sobre a carga cognitiva empregada no desenvolvimento de apresentações multimídia utilizando o método proposto e ferramentas de autoria tradicionais. \citeonline{kuccuk2016learning} realizam um estudo sobre a carga cognitiva empregada no estudo de anatomia com e sem o uso de realidade aumentada. Nesse trabalho, os autores dividem os estudantes em um grupo de controle e um grupo de experimento, em que o primeiro assiste à uma aula tradicional com o apoio de um livro e o segundo grupo assiste à aula com o auxílio de um aplicativo móvel de realidade aumentada além do mesmo livro utilizado pelo primeiro grupo. O grupo de experimento teve uma menor carga cognitiva além de ter alcançado melhores notas no exame de avaliação. Um estudo similar a este poderia ser conduzido para avaliar a carga cognitiva no desenvolvimento de apresentações multimídia com e sem o uso de realidade aumentada.

Finalmente, apesar do estudo qualitativo baseado no modelo TAM ter ajudado no alcance de conclusões significativas à respeito da abordagem BumbAR, ainda é necessário um estudo mais aprofundado, especialmente a comparando com outras ferramentas de autoria.

\section{Produção científica}

O presente trabalho gerou o artigo científico intitulado~"Exploring an AR-based User Interface for Authoring Multimedia Presentations"~\cite{mendes2018exploring} que foi aceito e publicado no principal congresso de engenharia de documentos, o "The ACM Symposium on Document Engineering (DocEng)"~realizado no Canadá no ano de 2018. 

\end{document}