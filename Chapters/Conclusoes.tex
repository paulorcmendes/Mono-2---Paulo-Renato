\documentclass[../main.tex]{subfiles}

\begin{document}

\chapter{Conclusões e Discussão}\label{cap:conclusoes}

Este trabalho propôs uma abordagem baseada em realidade aumentada para o desenvolvimento de apresentações multimídia. A abordagem proposta é baseada no modelo NCM, mais especificamente em seus elos causais, condições e ações. Além disso, foi apresentada a ferramenta BumbAR, que implementa a abordagem proposta com uma implementação baseada no Unity 3D, permitindo que usuários criem, visualizem, e gerem um documento NCL de suas apresentações multimídia. Para avaliar a proposta deste trabalho, foi desenvolvido um estudo qualitativo baseado no modelo TAM. Os participantes do estudo consideraram que a proposta desenvolvida é tanto útil quanto fácil de usar. Tal estudo também permitiu a evolução da proposta, evidenciando a necessidade da criação de algo que permitisse a visualização da apresentação sendo criada como um todo.

Apesar da implementação atual da ferramenta BumbAR discutida na Seção~\ref{sec:implementacao_prototipo} permitir o desenvolvimento de uma grande variedade de apresentações multimídia, ela ainda não suporta completamente todas as condições e ações presentes no modelo NCM, como as relacionadas a seleções e atribuições, por exemplo. Essas condições e ações permitem a modificação de maneira dinâmica das propriedades de objetos de mídia em uma apresentação multimídia, como seu volume e transparência. A implementação de outras condições e ações é deixada com um trabalho futuro.

Como mencionado anteriormente, a abordagem proposta utilizou os \emph{elos causais} do modelo NCM para especificar as relações e o sincronismo temporal entre objetos de mídia. Como outro trabalho futuro, pretende-se explorar outras abordagens para especificar tal sincronismo. Por exemplo, a abordagem da linguagem de integração multimídia sincronizada, do inglês \emph{Synchronized Multimedia Integration Language}(SMIL)~\cite{rutledge2001smil}, propõe o sincronismo entre objetos de mídia com base em dois conceitos: \emph{par} e \emph{seq}. Objetos de mídia em \emph{seq}(sequência) tocam sequencialmente: quando um acaba o próximo começa; enquanto objetos de mídia em \emph{par}~(paralelo) tocam simultaneamente: eles começam e terminam ao mesmo tempo. Para alguns tipos de apresentações multimídia, como uma apresentação de slides, é possível facilitar o processo de autoria utilizando tais abstrações.

Finalmente, apesar do estudo qualitativo baseado no modelo TAM ter ajudado no alcance de conclusões significativas à respeito da abordagem BumbAR, ainda é necessário um estudo mais aprofundado, especialmente a comparando com outras ferramentas de autoria.

\section{Produção científica}

O presente trabalho gerou o artigo científico "Exploring an AR-based User Interface for Authoring Multimedia Presentations"~\cite{mendes2018exploring} que foi aceito e publicado no principal congresso de engenharia de documentos, o "The ACM Symposium on Document Engineering (DocEng)"~realizado no Canadá no ano de 2018. 

\end{document}