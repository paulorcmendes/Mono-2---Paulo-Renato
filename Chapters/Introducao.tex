\documentclass[../main.tex]{subfiles}

\begin{document}

\chapter{Introdução}\label{cap:introducao}


Segundo \citeonline{jacob2008reality}, nas últimas décadas, pesquisadores da área de Interface Humano Computador~(IHC) tem desenvolvido diversos novos tipos de interface que divergem do modo mais comum de interação conhecido como WIMP~(que envolve janelas, ícones, menus e um dispositivo apontador). Esses novos modelos de interface são conhecidas como interfaces pós-WIMP, do inglês \emph{post-WIMP interfaces}. Essas propostas foram impulsionadas pelo avanço da tecnologia  e pelo maior entendimento da psicologia humana, o que proporcionou interfaces cada vez mais inovadoras. Definidas por \citeonline{van1997post} como interfaces que contém pelo menos uma técnica de interação não dependente de itens bidimensionais clássicos como menus e ícones, alguns exemplos de estilos de interação pós-WIMP são os seguintes: realidade virtual, mista, e aumentada, interação tangível, computação sensível ao contexto, computação ubíqua.

A realidade aumentada (RA) é definida por \citeonline{carmigniani2011augmented} como uma visão direta ou indireta em tempo real de um elemento físico do mundo real que foi aumentado através da adição de informação virtual a ele. Essas informações podem ser imagens, áudios, vídeos, toque, ou sensações hápticas sobrepostas em tempo real \cite{buchmann2004fingartips}. A realidade aumentada difere de outras técnicas, como realidade virtual (RV), onde a imersão do usuário em um um ambiente virtual é completa, impossibilitando a visão do mundo real. A Realidade Aumentada permite que o usuário enxergue o mundo real com a sobreposição de informações virtuais a ele. Assim, os sistemas que fazem uso dessa técnica complementam a realidade, em vez de a substituir. De fato, em anos recentes, a realidade aumentada vem sendo utilizada em diversos campos em razão de sua capacidade de auxiliar usuários na realização de tarefas do mundo real, contribuindo para a melhora de seu desempenho. 


%balog2010role mostrou que que a utilidade percebida e o engajamento percebido possuem um impacto significante na intenção comportamental de usar o ARTP(), enquanto a facilidade não é tão significante.

Além disso, o uso de RA tem demonstrado estar diretamente relacionado à satisfação e engajamento do usuário na condução de tarefas. O trabalho de \citeonline{balog2010role} realizou um estudo à respeito de uma plataforma de ensino que utiliza RA. Tal estudo mostrou que o engajamento percebido se mostrou um fator influenciador na chave na intenção de uso em relação à plataforma. De modo similar, o trabalho de \citeonline{wojciechowski2013evaluation} avaliou um sistema para criação e apresentação de ambientes de aprendizagem de realidade aumentada baseadas em imagens 3D e concluiu que o engajamento possuiu um papel dominante na intenção dos usuários em utilizar o sistema.

%\cite{haugstvedt2012mobile, wojciechowski2013evaluation}.


Ferramentas de autoria multimídia suportam o desenvolvimento de produtos multimídia (como vídeos interativos e apresentações) através da integração de objetos de mídia de diferentes tipos (como imagens, sons e vídeos) de uma maneira sincronizada e com significado. Apesar da qualidade de um conteúdo multimídia ser consideravelmente dependente da qualidade individual dos objetos de mídia que o compõe, é somente com uma composição apropriada de todos esses objetos em uma apresentação que os usuários finais irão, verdadeiramente, ter uma experiência de qualidade \cite{pellan2009authoring}. 

Uma maneira comum de representar apresentações multimídia interativas é através de documentos multimídia, como a linguagem de marcação de hipertexto, do inglês \emph{Hypertext Markup Language} (HTML) \cite{graham1995html}, a 
linguagem de integração multimídia sincronizada, do inglês \emph{Synchronized Multimedia Integration Language} (SMIL) \cite{rutledge2001smil}, e a 
linguagem de contexto aninhado, do inglês \emph{Nested Context Language} (NCL) \cite{soares2009programando}. A maioria dos documentos de multimídia atuais fazem uma separação clara da especificação dos objetos de mídia que compões a apresentação multimídia e da especificação do sincronismo entre esses objetos de mídia, que define a dependência temporal entre eles. Existem algumas diferenças na forma como é especificado esse sincronismo nos diferentes modelos de documento multimídia. Alguns deles, como NCL e HTML, explicitam uma entidade chamada de \emph{elo} (do inglês \emph{link}). Em outros casos, como nos documentos da linguagem SMIL, os objetos de mídia são organizados em estruturas hierárquicas, chamadas de composições temporais, que implicitamente definem a sincronização de objetos de mídia. Além disso, alguns desses documentos suportam outros tipos de composições (por exemplo, os elementos \emph{context} e \emph{div} em NCL e HTML, respectivamente) como uma forma de simplificar a autoria e estrutura de documentos complexos. De qualquer forma, a especificação do sincronismo é uma das tarefas que mais demanda tempo no processo de autoria multimídia, o que pode influenciar negativamente o engajamento de usuários na condução de tal tarefa.

Apesar dos conceitos usados nos modelos de documentos supracitados serem de fácil compreensão, a representação textual deles não é de fácil utilização por não-programadores, que geralmente estão mais familiarizados com representações visuais. Ferramentas de autoria multimídia são programas que permitem a concepção de apresentações multimídia com o objetivo de simplificar o processo de autoria, permitindo que não-programadores possam criar tais apresentações. De fato, o desenvolvimento de um método que simplifique o processo de autoria deve aumentar a satisfação do usuário em desenvolver apresentações multimídia, abrindo novas possibilidades para pessoas criativas criarem experiências multimídia. Como, atualmente, a forma mais comum de interação com um computador é ainda através de uma interface gráfica do usuário (do inglês \emph{Graphical User Interface (GUI)}), com o uso de janelas, ícones, menus e tipicamente um mouse, grande parte das ferramentas de autoria multimídia atuais adotam esse paradigma.

Diferente dessas ferramentas, este trabalho propõe um processo de autoria multimídia baseado em realidade aumentada. Nesse processo, ao invés do uso de uma GUI comum, é utilizada a realidade aumentada (RA) como uma forma inovadora de interface com o usuário.  

\section{Objetivos}
    Esta seção detalha os objetivos geral e específicos do presente trabalho.
\subsection{Objetivo geral}
    O principal objetivo deste trabalho é a proposição e avaliação de um processo de autoria multimídia que faça uso de realidade aumentada como forma de interface com o usuário.
    
\subsection{Objetivos específicos}

\prcnote{Pensar em outros objetivos específicos}
\begin{itemize}
    \item Desenvolver uma ferramenta que utilize o processo proposto, propiciando a autoria de apresentações multimídia por meio da realidade aumentada.
    
    \item Avaliar o processo proposto com usuários.
\end{itemize}

\section{Organização do trabalho}

Além do Capítulo \ref{cap:introducao}, este trabalho está organizado em mais 4 capítulos, sendo que o Capítulo \ref{cap:relacionados} apresenta os trabalhos relacionados a este.


O Capítulo \ref{cap:proposta} descreve o proposta deste trabalho. Tal capítulo descreve o processo de autoria proposto, bem como sua implementação na ferramenta de autoria desenvolvida.

O Capítulo \ref{cap:avaliacao} detalha a forma como o método proposto foi avaliado, apresenta e discute os resultados obtidos.

Por fim, o Capítulo \ref{cap:avaliacao} contém as conclusões e considerações finais deste trabalho.


\end{document}