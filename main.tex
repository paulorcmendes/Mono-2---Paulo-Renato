%% abtex2-modelo-relatorio-tecnico.tex, v-1.9.2 laurocesar
%% Copyright 2012-2014 by abnTeX2 group at http://abntex2.googlecode.com/ 
%%
%% This work may be distributed and/or modified under the
%% conditions of the LaTeX Project Public License, either version 1.3
%% of this license or (at your option) any later version.
%% The latest version of this license is in
%%   http://www.latex-project.org/lppl.txt
%% and version 1.3 or later is part of all distributions of LaTeX
%% version 2005/12/01 or later.
%%
%% This work has the LPPL maintenance status `maintained'.
%% 
%% The Current Maintainer of this work is the abnTeX2 team, led
%% by Lauro César Araujo. Further information are available on 
%% http://abntex2.googlecode.com/
%%
%% This work consists of the files abntex2-modelo-relatorio-tecnico.tex,
%% abntex2-modelo-include-comandos and abntex2-modelo-references.bib
%%

% ------------------------------------------------------------------------
% ------------------------------------------------------------------------
% abnTeX2: Modelo de Relatório Técnico/Acadêmico em conformidade com 
% ABNT NBR 10719:2011 Informação e documentação - Relatório técnico e/ou
% científico - Apresentação
% ------------------------------------------------------------------------ 
% ------------------------------------------------------------------------

\documentclass[
	% -- opções da classe memoir --
	12pt,				% tamanho da fonte
	openright,			% capítulos começam em pág ímpar (insere página vazia caso preciso)
	oneside,			% para impressão em verso e anverso. Oposto a oneside
	a4paper,			% taman
	ho do papel. 
	% -- opções da classe abntex2 --
	%chapter=TITLE,		% títulos de capítulos convertidos em letras maiúsculas
	%section=TITLE,		% títulos de seções convertidos em letras maiúsculas
	%subsection=TITLE,	% títulos de subseções convertidos em letras maiúsculas
	%subsubsection=TITLE,% títulos de subsubseções convertidos em letras maiúsculas
	% -- opções do pacote babel --
	english,			% idioma adicional para hifenização
	french,				% idioma adicional para hifenização
	spanish,			% idioma adicional para hifenização
	brazil,				% o último idioma é o principal do documento
	]{abntex2}

% ---
% Pacotes básicos 
% ---
\usepackage{lmodern}
% Usa a fonte Latin Modern	

\usepackage[T1]{fontenc}
%\hyphenation{ ir-re-ver-s\'{i}-veis}
% Selecao de codigos de fonte.
\usepackage[utf8]{inputenc}			% Codificacao do documento (conversão automática dos acentos)
\usepackage{lastpage}				% Usado pela Ficha catalográfica
\usepackage{indentfirst}			% Indenta o primeiro parágrafo de cada seção.
\usepackage{color}					% Controle das cores
\usepackage{graphicx}				% Inclusão de gráficos e imagens
\usepackage{microtype} 				% para melhorias de justificação
\usepackage{mathtools}				% Matemática
\usepackage{multirow}				% Tabelas
\usepackage[table,xcdraw]{xcolor}	% Tabelas (cores)
% ---
\usepackage{float}
\usepackage{gensymb}

\usepackage{amssymb}
\usepackage{subcaption}
\usepackage[utf8]{inputenc}
\usepackage[export]{adjustbox}
\usepackage{wrapfig}
\usepackage{amsmath}
\usepackage{mathtools}
\usepackage{pdfpages}
\usepackage{subfiles}

\graphicspath{{figs/}}

\hyphenpenalty=10000
		
% ---
% Pacotes adicionais, usados apenas no âmbito do Modelo Canônico do abnteX2
% ---
\usepackage{lipsum}				% para geração de dummy text
% ---

% ---
% Pacotes de citações
% ---
\usepackage[brazilian,hyperpageref]{backref}	 % Paginas com as citações na bibl
%\usepackage[alf]{abntex2cite}	% Citações padrão ABNT
\usepackage[alf,abnt-etal-list=0]{abntex2cite}

\usepackage{tabularx}

\newcommand{\otoprule}{\midrule[\heavyrulewidth]}
\usepackage{color,soul}
\usepackage[nomargin,inline]{fixme}
\fxusetheme{color}
\fxsetup{draft}
% \fxsetup{final}
\FXRegisterAuthor{prc}{pm}{\color{red}Paulo}
\FXRegisterAuthor{cls}{cs}{\color{red}Salles}
% --- 
% CONFIGURAÇÕES DE PACOTES
% --- 

% ---
% Configurações do pacote backref
% Usado sem a opção hyperpageref de backref
\renewcommand{\backrefpagesname}{Citado na(s) página(s):~}
% Texto padrão antes do número das páginas
\renewcommand{\backref}{}
% Define os textos da citação
\renewcommand*{\backrefalt}[4]{
	\ifcase #1 %
		Nenhuma citação no texto.%
	\or
		Citado na página #2.%
	\else
		Citado #1 vezes nas páginas #2.%
	\fi}%
% ---

% ---
% Informações de dados para CAPA e FOLHA DE ROSTO
% ---
\titulo{Autoria Multimídia com Uso de Realidade Aumentada}
\autor{Paulo Renato Conceição Mendes}
\local{São Luís - MA}
\data{2019}
\orientador{Prof. Dr. Carlos de Salles Soares Neto}
\tipotrabalho{Monografia (Graduação)}
% O preambulo deve conter o tipo do trabalho, o objetivo, 
% o nome da instituição e a área de concentração 
\preambulo{Monografia apresentada ao curso de Ciência da Computação da Universidade Federal do Maranhão como parte dos requisitos necessários para obtenção do grau de bacharel em Ciência da Computação.}
% ---


% ---
% Configurações de aparência do PDF final

% alterando o aspecto da cor azul
\definecolor{blue}{RGB}{41,5,195}

% informações do PDF
\makeatletter
\hypersetup{
     	%pagebackref=true,
		pdftitle={\@title}, 
		pdfauthor={\@author},
    	pdfsubject={\imprimirpreambulo},
	    pdfcreator={LaTeX with abnTeX2},
		pdfkeywords={abnt}{latex}{abntex}{abntex2}{trabalho acadêmico}, 
		colorlinks=true,       		% false: boxed links; true: colored links
    	linkcolor=blue,          	% color of internal links
    	citecolor=blue,        		% color of links to bibliography
    	filecolor=magenta,      		% color of file links
		urlcolor=blue,
		bookmarksdepth=4
}
\makeatother
% --- 

% --- 
% Espaçamentos entre linhas e parágrafos 
% --- 

% O tamanho do parágrafo é dado por:
\setlength{\parindent}{1.3cm}

% Controle do espaçamento entre um parágrafo e outro:
\setlength{\parskip}{0.2cm}  % tente também \onelineskip

% ---
% compila o indice
% ---
\makeindex
% ---

% ----
% Início do documento
% ----
\begin{document}

% Seleciona o idioma do documento (conforme pacotes do babel)
%\selectlanguage{english}
\selectlanguage{brazil}

% Retira espaço extra obsoleto entre as frases.
\frenchspacing 

% ----------------------------------------------------------
% ELEMENTOS PRÉ-TEXTUAIS
% ----------------------------------------------------------
% \pretextual

% ---
% Capa
% ---
\imprimircapa
% ---CDU 
%\includepdf{ficha.pdf}
% ---
% Folha de rosto
% (o * indica que haverá a ficha bibliográfica)
% ---
%\imprimirfolhaderosto
\begin{folhadeaprovacao}


 	\begin{center}
 		{\ABNTEXchapterfont\large\imprimirautor} \\
         \vspace{1cm} 
 		{\ABNTEXchapterfont\Large\bfseries\imprimirtitulo}
 	\end{center}
   \vspace{1cm} 
   \hspace{.45\textwidth} 
   \begin{minipage}{.5\textwidth} 
   \imprimirpreambulo \vspace{1cm} 
 	\end{minipage} 
 	\vspace{1cm}  \\
 	Trabalho aprovado em 30 de maio de 2019:
      %\imprimirlocal, \imprimirdata: 
 %%%%%%%%%%%%%%%%%%%%%%%%%%%%%%%%%%%%%%%%%%%%%%%% 
 %Assinaturas 
% %%%%%%%%%%%%%%%%%%%%%%%%%%%%%%%%%%%%%%%%%%%%%% 
 \assinatura{\imprimirorientador \\ Orientador \\ Universidade Federal do Maranhão } 


 \assinatura{Prof. Dr. Examinador1 \\Examinador\\Universidade Federal do Maranhão} 
 \assinatura{Prof. Dr. Examinador2\\ Examinadora\\ Universidade Federal do Maranhão} 

% %%%%%%%%%%%%%%%%%%%%%%%%%%%%%%%%%%%%%%%%%%%%%%%%%% 
% % 
% %%%%%%%%%%%%%%%%%%%%%%%%%%%%%%%%%%%%%%%%%%%%%%%%%%% 
 \begin{center} 
 \vfill 
 {\large\imprimirlocal} 
 \par 
 {\large\imprimirdata} 
 \end{center} 
 \end{folhadeaprovacao} 

%\includepdf{assinaturas.pdf}

%%%%%%%%%%%%%%%%%%%%%%%%%%%%%%%%% 
% Início da dedicatória - Elemento opcional 
%%%%%%%%%%%%%%%%%%%%%%%%%%%%%%%%%%%%%%%%%%%%%%%%%%%%%%%%%%% 
\begin{dedicatoria} 
\vspace{\fill} 
\begin{center} 
À minha família e meus amigos.
\end{center}
\vspace{\fill} 
\end{dedicatoria} 
%%%%%%%%%%%%%%%%%%%%%%%%%%%%%%%%%%%%%%%%%%%%%%%%%% 
% Fim da dedicatória 
%%%%%%%%%%%%%%%%%%%%%%%%%%%%%%%%%%%%%%%%%%%%%%%%%%


% ---
% Anverso da folha de rosto:
% ---



%\ABNTEXchapterfont

%\vspace*{\fill}

%Conforme a ABNT NBR 10719:2011, seção 4.2.1.1.1, o anverso da folha de rosto
%deve conter:
%
% \begin{alineas}
%  \item nome do órgão ou entidade responsável que solicitou ou gerou o
%   relatório; 
%  \item título do projeto, programa ou plano que o relatório está relacionado;
%  \item título do relatório;
%  \item subtítulo, se houver, deve ser precedido de dois pontos, evidenciando a
%   sua subordinação ao título. O relatório em vários volumes deve ter um título
%   geral. Além deste, cada volume pode ter um título específico; 
%  \item número do volume, se houver mais de um, deve constar em cada folha de
%   rosto a especificação do respectivo volume, em algarismo arábico; 
%  \item código de identificação, se houver, recomenda-se que seja formado
%   pela sigla da instituição, indicação da categoria do relatório, data,
%   indicação do assunto e número sequencial do relatório na série; 
%  \item classificação de segurança. Todos os órgãos, privados ou públicos, que
%   desenvolvam pesquisa de interesse nacional de conteúdo sigiloso, devem
%    informar a classificação adequada, conforme a legislação em vigor; 
%  \item nome do autor ou autor-entidade. O título e a qualificação ou a função
%   do autor podem ser incluídos, pois servem para indicar sua autoridade no
%   assunto. Caso a instituição que solicitou o relatório seja a mesma que o
%   gerou, suprime-se o nome da instituição no campo de autoria; 
%  \item local (cidade) da instituição responsável e/ou solicitante; NOTA: No
%   caso de cidades homônimas, recomenda-se o acréscimo da sigla da unidade da
%   federação.
%  \item ano de publicação, de acordo com o calendário universal (gregoriano),
%  deve ser apresentado em algarismos arábicos.
% \end{alineas}

%\vspace*{\fill}
%}

% ---
% Agradecimentos
% ---
\begin{agradecimentos}

Aos meus pais, Antonio e Dulce, por me ensinaram desde cedo sobre o poder transformador da educação, pelo apoio de sempre e por me guiarem durante toda a minha formação.

À minha irmã, Nicole, pelo incentivo de sempre mesmo eu a irritando constantemente.

Ao meu orientador, Carlos de Salles, pelos conselhos, orientações, conversas e apoio desde o começo do curso.

Aos professores Simara, Ari, Anselmo, Geraldo, Giovanni, Dallyson e todos os demais professores que contribuiram para a minha formação.

A Ninguém (Pedro Almeida), por ser o companheiro de todas as horas e grande parceiro nessa jornada acadêmica.

A Erik e Robert, pelas caronas e amizade.

A Luciano, pelas conversas e conselhos.

A Ylderlan, por ser um grande exemplo de esforço e dedicação.

A Polyana, pela amizade e conselhos acadêmicos e não acadêmicos.

A Anderson Silva, por ser um grande amigo desde o ensino médio.

A todos os demais amigos do codebuilders, companheiros nessa jornada de conhecimentos e memes.

A Lucas, Daniel, Ruy e todos os demais amigos e parceiros do Telemídia-MA.

A Kelson, Ricardo, Arthur, Eduardo e todos os demais colegas do LabPai.

A Roberto Gerson, por sua ajuda e orientação na condução deste trabalho.

A Fabiana Moura, cujo apoio foi decisivo para o alcance que este trabalho obteve.

A Márcio Moreno e Jack Jansen, cujos comentários ajudaram a evoluir este trabalho.

À FAPEMA, pelo apoio financeiro a este trabalho.



\end{agradecimentos}

%%%%%%%%%%%%%%%%%%%%%%% 
% Início da epígrafe - opcional 
%%%%%%%%%%%%%%%%%%%%%%%%%%%%%%%%%%%%%%%%%%%%%%%%%%%%%%%%%%%%%% 
\begin{epigrafe} 
\vspace*{\fill} 
\begin{flushright} 
\textit{"São as nossas escolhas, mais do que as nossas capacidades, que mostram quem realmente somos."} 

\textit{(Alvo Dumbledore)}
\end{flushright} 


\end{epigrafe} 
%%%%%%%%%%%%%%%%%%%%%%%%%%%%%%%%%%%%%%%%%%%%%%%%%%%%%%%%%%%%%%%%% 
% Fim da epígrafe - opcional 
%%%%%%%%%%%%%%%%%%%%%%%%%%%%%%%%%%%%%%%%%%%%%%%%%%%%%%%%%%%%%%

% ---

% ---
% RESUMO
% ---

% resumo na língua vernácula (obrigatório)
%\setlength{\absparsep}{18pt} % ajusta o espaçamento dos parágrafos do resumo
\begin{resumo}

Este trabalho apresenta a BumbAR, uma abordagem baseada em realidade aumentada para a autoria de apresentações multimídia e avalia tal abordagem por meio de um estudo qualitativo baseado no modelo TAM. A abordagem BumbAR é baseada no modelo NCM e explora o uso de realidade aumentada e objetos do mundo real (marcadores) como uma forma inovadora de interface com o usuário para descrever os comportamentos e relações entre objetos de mídia presentes na apresentação. O estudo qualitativo teve o objetivo de avaliar a atitude dos usuários em relação ao uso da BumbAR. Os resultados mostraram que os participantes consideram que a abordagem proposta é útil e fácil de usar, enquanto a maioria deles consideram que o sistema é mais conveniente do que ferramentas de autoria tradicionais. Os comentários dos participantes do estudo mostraram a necessidade da inclusão de novas funcionalidades na ferramenta. 

\noindent
\textbf{Palavras-chaves}: Realidade aumentada, multimídia, ferramenta de autoria, interface de usuário.

\end{resumo}

% resumo em inglês
% \include{estrutura/resumo_en}
\begin{resumo}[Abstract]
\begin{otherlanguage*}{english}

This work presents BumbAR, an approach based on augmented reality for authoring multimedia presentations and evaluate it through a qualitative study based on the TAM model. The BumbAR approach is based on the NCM model and explores the use of augmented reality and real-world objects (markers) as an innovative user interface for describing the behavior and relationships between the media objects that are part of a multimedia presentation. The qualitative study aimed at evaluating users' attitude towards using BumbAR. The results showed that the participants found that the proposed approach is both useful and easy-to-use, while most of them found the system more convenient than traditional desktop-based authoring tools.

\noindent
\textbf{Keywords}: Augmented reality, multimedia, authoring tool, user interface.
\end{otherlanguage*}
\end{resumo}
% ---
\cleardoublepage
% ---
% inserir lista de ilustrações
% ---
\pdfbookmark[0]{\listfigurename}{lof}
\listoffigures*
\cleardoublepage
% ---

% ---
% inserir lista de tabelas
% ---
\pdfbookmark[0]{\listtablename}{lot}
\listoftables*
\cleardoublepage
% ---

% ---
% inserir lista de abreviaturas e siglas
% ---
\begin{siglas}
    %\item[AACR] American Association for Cancer Research

  \item[GUI] Graphical User Interface
  \item[HTML] Hypertext Markup Language
  \item[IHC] Interação Humano Computador
  \item[NCL] Nested Context Language  
  \item[NCM] Nested Context Model  
  \item[PEOU] Perceived Ease-of-use
  \item[POI] Points of Interest    
  \item[PU] Perceived Usefulness
  \item[RA] Realidade Aumentada
  \item[RV] Realidade Virtual  
  \item[SMIL] Synchronized Multimedia Language  
  \item[TAM] Technology Acceptance Model
  \item[TUI] Tangible User Interface
  \item[URL] Uniform Resource Locator
  
  
  
\end{siglas}
% ---

% ---
% inserir lista de símbolos
% ---
%\begin{simbolos}
%  \item[$ \in $] Pertence
%\end{simbolos}
% ---

% ---
% inserir o sumario
% ---
\pdfbookmark[0]{\contentsname}{toc}
\tableofcontents*
\cleardoublepage
% ---


% ----------------------------------------------------------
% ELEMENTOS TEXTUAIS
% ----------------------------------------------------------
\textual

\subfile{Chapters/Introducao.tex} 

\subfile{Chapters/Relacionados.tex} 

\subfile{Chapters/Proposta.tex} 

\subfile{Chapters/Avaliacao.tex} 

\subfile{Chapters/Conclusoes.tex} 

\postextual

% ----------------------------------------------------------
% Referências bibliográficas
% ----------------------------------------------------------
\bibliography{main.bib}
\end{document}